%\documentclass{ltjsarticle}
\documentclass[../main]{subfiles}
%\usepackage{listings}%,jlisting}

%\usepackage{url} %URLをリンクとして表示するためのパッケージ


\lstset{
    basicstyle={\ttfamily\small}, %書体の指定
    frame=tRBl, %フレームの指定
    framesep=10pt, %フレームと中身(コード)の間隔
    breaklines=true, %行が長くなった場合の改行
    linewidth=12cm, %フレームの横幅
    lineskip=-0.5ex, %行間の調整
    tabsize=2 %Tabを何文字幅にするかの指定
}


%\title{サンプルレポート}
%\author{MAST編集部}


\begin{document}
%\maketitle


\section{プログラミング}
プログラミングをしてみましょう。
\subsection{大量のhoge}
“hoge”とは,プログラミングなどで,「意味のないこと」を表現するために利用されることが多い文字列である。
情報処理の世界ではメタ構文変数(Metasyntactic variable)
とも呼ばれる\footnote{hogeとは – はてなキーワード: \url{http://d.hatena.ne.jp/keyword/hoge}}。\\


for文を使わずに“hoge”を大量に表示するプログラムを作ろう。


\begin{itemize}
\item ソースコード
\begin{lstlisting}
#include 


int main(void){
    printf(“hoge\n”); //hogeのはじまり
    printf(“hoge\n”);
    printf(“hoge\n”);
    printf(“(省略)\n”);
    printf(“hoge\n”);
    printf(“hoge\n”);


    return 0;
}
\end{lstlisting}


\item 実行結果
\begin{lstlisting}
$ cc hoge.c
$ ./a.out
hoge
hoge
hoge
(省略)
hoge
hoge
$
\end{lstlisting}
\end{itemize}


\end{document}